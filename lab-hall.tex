\input{text/diss}

\begin{document}

\def\labauthors{Понур К.А., Сарафанов Ф.Г., Сидоров Д.А.}
\def\labgroup{420}
\def\labnumber{000}
\def\labtheme{Эффект Холла}

\input{text/titlepage}

\tableofcontents
\newpage

\section*{Введение}
\addcontentsline{toc}{section}{Введение}
\label{sec:input}

В данной работе исследуется эффект Холла.

Эффект  Холла  представляет собой  в появление   поперечной    э.д.с.    при    прохождении 
электрического  тока  через  проводник,   помещенный   в   магнитное   поле, перпендикулярное к направлению тока. 

Измерение  холловской разности потенциалов  обычно  позволяет  определить  концентрацию  и знак основных носителей заряда в веществе.

Целью данной работы является изучение возникновения эффекта Холла в слабом магнитном поле, определение коэффициента Холла, холловской подвижности, определение концентрации основных носителей в образце.



\end{document}