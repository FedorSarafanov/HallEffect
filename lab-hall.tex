\input{text/diss}

\begin{document}

\def\labauthors{Понур К.А., Сарафанов Ф.Г., Сидоров Д.А.}
\def\labgroup{420}
\def\labnumber{204}
\def\labtheme{Эффект Холла}
\renewcommand{\vec}{\mathbf}
\input{text/titlepage}

\tableofcontents
\newpage

\section*{Введение}
\addcontentsline{toc}{section}{Введение}
\label{sec:input}

В данной работе исследуется эффект Холла.

Эффект Холла представляет собой в появление поперечной э.д.с. при прохождении 
электрического тока через проводник, помещенный в магнитное поле, перпендикулярное к направлению тока. 

Измерение холловской разности потенциалов обычно позволяет определить концентрацию и знак основных носителей заряда в веществе.

Целью данной работы является изучение возникновения эффекта Холла в слабом магнитном поле, определение коэффициента Холла, холловской подвижности, определение концентрации основных носителей в образце.

\section{Анализ теории}
\subsection[Описание эффекта Холла]{Описание эффекта Холла без учёта механизма рассеяния носителей заряда}
\subsubsection{Разделение зарядов}
Рассмотрим образец, через который протекает ток $\vec{j}$.
\begin{figure}[H]
	\centering
	\includegraphics[width=0.85\textwidth]{img/effect}
	\caption{Механизм разделения зарядов при проявлении эффекта Холла}
	\label{fig:figure1}
\end{figure}

Электрическое поле $\vec{Е}$ создает в полупроводнике электрический ток плотностью 
\begin{equation}
	\vec{j}=\sigma\vec{E},
\end{equation}
где $\sigma=\frac{1}{\rho}$ -- удельная электрическая проводимость, $\rho$ -- удельное сопротивление проводника. 

Со стороны магнитного поля В на движущиеся заряды действует магнитная составляющая силы Лоренца 
\begin{equation}
	\label{eq:fl}
	\vec{F}_L=q[\vec{v}\times\vec{B}]
\end{equation}
Здесь $\vec{v}$ - дрейфовая скорость носителей заряда.

Под действием этой силы происходит разделение зарядов на противоположных боковых (параллельных току и магнитному полю) гранях образца.

При разделении зарядов грани заряжаются, и возникает поперечное поле $\vec{E}_\perp$ -- поле Холла. На заряд начинает действовать сила Кулона
\begin{equation}
	\vec{F}_K=q\vec{E}_\perp
\end{equation}

\subsubsection{Равновесное состояние}

Поле Холла препятствует движению зарядов, вызванным действием магнитного поля, и на некотором этапе разделения зарядов наступает равновесие сил $\vec{F}_K$ и $\vec{F}_L$:
\begin{equation}
	\label{eq:eq}
	q\vec{E}_\perp+\vec{F}_L=0
\end{equation}

Отсюда
\begin{equation}
	\label{eq:ep}
	\vec{E}_\perp=-[\vec{v}\times\vec{B}]
\end{equation}

Обозначим плотность тока:
\begin{equation}
j=\frac{I}{S}
\end{equation}
Распишем ток:
\begin{equation}
I=\frac{dQ}{dt}=\frac{qnvS\,dt}{dt}=qnvS,
\end{equation}
где $n$ - концентрация заряда по объему.

Отсюда
\begin{equation}
v=\frac{I}{qnS}=\frac{j}{qn}=\frac{\sigma E}{qn}
\end{equation}
То есть
\begin{equation}
	\label{eq:mue}
	v\sim E
	\quad\Rightarrow\quad
	\vec{v}=\mu \vec E, \quad \mu=\frac{\sigma}{qn}
\end{equation}

Из (\ref{eq:ep}), (\ref{eq:mue}) следует
\begin{equation}
	\label{eq10}
	\vec{E}_\perp=-\mu[\vec{E}\times\vec{B}]
\end{equation}
Или с учетом $\vec{j}=\sigma\vec{E}$
\begin{equation}
	\vec{E}_\perp=-R[\vec{j}\times\vec{B}]
\end{equation}

Где $R$ -- коэффициент Холла.

\subsubsection{Коэффициент Холла}
В нашем выводе
\begin{equation}
	R\sigma=\mu
	\quad \Rightarrow\quad
	R=\frac{1}{qn}
\end{equation}

При более строгом выводе, учитывающем механизм рассеяния свободных носителей заряда, можно получить 
\begin{equation}
	R=\frac{\gamma}{qn}
\end{equation}

Где $\gamma$ -- холл-фактор, безразмерный коэффициент, зависящий от величины магнитного поля и механизма рассеяния свободных носителей заряда при их взаимодействии с ионами примесей и кристаллической решеткой. 

Для используемого в данной лабораторной работе чистого слабо легированного германия при комнатной температуре в слабом магнитном поле $\gamma\approx 1.18$.

\subsubsection{Холловская подвижность}

Произведение $R\sigma$ имеет размерность подвижности и называется холловской подвижностью:
\begin{equation}
	\mu_H=R\sigma
\end{equation}

\subsubsection{Угол Холла}

Действие магнитного поля $\vec{B}$ приводит к тому, что суммарное электрическое поле 
\begin{equation}
	\vec{E}_\Sigma=\vec{E}+\vec{E}_\perp
\end{equation}
оказывается повернутым на некоторый угол $\vartheta$ (угол Холла) относительно вектора плотности тока. Из полученных ранее выражений можно показать, что
\begin{equation}
	\tan\vartheta=-\mu_H B
\end{equation}
При слабом магнитном поле 
\begin{equation}
	-\mu_H B \ll 1
\end{equation}
 угол Холла приближенно можно вычислить по формуле 
\begin{equation}
	\vartheta=-\mu_H B
\end{equation}

\subsubsection{Холловская разность потенциалов}

Эквипотенциальные поверхности в средней части ограниченного вытянутого образца поворачиваются при включении магнитного поля В на угол $\vartheta$ относительно их первоначального положения.

Из-за этого в точках, изначально лежащих на эквипотенциали, появляется разность потенциалов $U_H$, называемая холловской разностью потенциалов.

Для образца прямоугольной формы в приближении однородного поля Холла эта разность потенциалов будет равна 
\begin{equation}
	U_H=bE_\perp
\end{equation}

Для прямоугольного образца
\begin{equation}
	j=\frac{I}{S}=\frac{I}{bc}
\end{equation}
Из (\ref{eq:mue}), (\ref{eq10}) следует
\begin{equation}
	{E}_\perp=\frac{j}{qn}B
\end{equation}
Откуда
\begin{equation}
	\frac{U_H}{b}=\frac{I}{qn\cdot bc}B
\end{equation}
И окончательно
\begin{equation}
	U_H=\frac{R}{c}IB
\end{equation}

\subsection{Побочные факторы}
\subsubsection{Нехолловская составляющая измеряемой разности потенциалов}

При изготовлении образца не удается разместить оба холловских контакта таким образом, чтобы они в отсутствие магнитного поля лежали на одной эквипотенциальной поверхности. 

В реальном образце между плоскостями расположения контактов всегда есть небольшое смещение $\Delta x$. 

При $\vec{B}=0$ и $I\ne0$ между этими плоскостями устанавливается разность потенциалов, равная 
\begin{equation}
	U_{34}=R_{34}I,
	\text{  где  }
	R_{34}=\rho\frac{\Delta x}{bc}
\end{equation}

Другие побочные факторы дают вклад в разность потенциалов между контактами 3 и 4 существенно меньший холловской разности потенциалов. 

Таким образом, в рамках нашей модели справедливо выражение
\begin{equation}
	U_H=U_{34}|_{B\ne 0}-U_{34}|_{B= 0}=\frac{RIB}{c}
\end{equation}

Отсюда видно, что коэффициент Холла $К$ может быть определен по тангенсу угла наклона линейных участков экспериментально снятых зависимостей $U_{34}(B)|_{I=\const}$ и $U_H(I)|_{B=\const}$.

\section{Экспериментальные данные}
\subsection{Вольт-амперная характеристика участка $R_{56}$}
\begin{figure}[H]
	\centering
	\includegraphics[width=\textwidth]{img/vax}
	\caption{ВАХ элемента}
	\label{fig:vax}
\end{figure}
\end{document}