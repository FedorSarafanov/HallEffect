\input{text/diss}

\begin{document}

\def\labauthors{Понур К.А., Сарафанов Ф.Г., Сидоров Д.А.}
\def\labgroup{420}
\def\labnumber{000}
\def\labtheme{Эффект Холла}

\input{text/titlepage}

\tableofcontents
\newpage

\section*{Введение}
\addcontentsline{toc}{section}{Введение}
\label{sec:input}

\begin{figure}[H]
	\centering
	\includegraphics[]{img/fig1}
	\caption{Caption here}
	\label{fig:figure1}
\end{figure}

В данной работе исследуется эффект Холла.

Эффект  Холла  представляет собой  в появление   поперечной    э.д.с.    при    прохождении 
электрического  тока  через  проводник,   помещенный   в   магнитное   поле, перпендикулярное к направлению тока. 

Измерение  холловской разности потенциалов  обычно  позволяет  определить  концентрацию  и знак основных носителей заряда в веществе.

% Целью данной работы является 




\end{document}